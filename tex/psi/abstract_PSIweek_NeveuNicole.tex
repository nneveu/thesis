\documentclass{article}     % onecolumn (second format)
\usepackage{subfigure,tikz,fullpage}
\usepackage{mathptmx}      % use Times fonts if available on your TeX system

\newcommand{\jlnote}[1]{\textsf{{\color{violet}{ LS note:}   #1 }}}
\newcommand{\nnnote}[1]{\textsf{{\color{blue}{ NN note:}   #1 }}}
\newcommand{\jpnote}[1]{\textsf{{\color{green}{ JP note:}   #1 }}}

\begin{document}
	
\section{Photoinjector optimization using opt-pilot and OPAL-T}
	
The goal of the week at PSI would be to generate a Pareto front 
for a photoinjector using opt-pilot and OPAL-T. The multi-objective
problem of interest is based on a currently installed and operated 
beam line at the Argonne Wakefield Accelerator facility. The simulation
model includes a 1.5 cell rf gun, solenoids, 6 identical 
accelerating cavities, 4 quads, and 2 dipoles. 
There are 14 variables over which we will optimize: 
gun phase, laser radius, laser FWHM, solenoid strength, the phase 
of each accelerating cavity, and quad strengths. 
The objectives are emittance and bunch length.
If time permits, we would create Pareto fronts for several charge values.    
	
\end{document}