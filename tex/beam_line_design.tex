\Chapter{Beam Line Design}%________________________________
Using the foundations set in Chapters 2 and 3, 
\nrnote{Change foundations to more elaborate description...
beam line requirements, diagnostics at AWA, etc...???}
a beam line design for staged 
two beam acceleration was laid out and simulated. 
\nrnote{.....more intro here...do I need to spell out building block matrices?}
A drift: 
\begin{equation}
R_d = 
\begin{bmatrix}
1 & L \\
0 & 1
\end{bmatrix}
\end{equation}

A quad: 
\begin{equation}
R_q = 
\begin{bmatrix}
1 & 0 \\
\pm \frac{1}{f} & 1
\end{bmatrix}
\end{equation}

A dipole:
\begin{equation}
R_s = 
\begin{bmatrix}
0 & 0 \\
0 & 0
\end{bmatrix}
\end{equation}

\Section{Matrix Formalism For TBA Beam Line}
 To reduce the number of free parameters quickly, and without using expensive PIC simulations, 
 the transfer matrix of the beam line was considered. Starting at the end of the linac, 
 we consider the first four quadrupoles before the kicker. All the quadruple strengths (4) and
 distances between the quadrupoles (6) are parameters under consideration. To reduce the number
 of variables from 10 to 4, we use the telescope module as described by K. Brown in \cite{brown}.   
 The transfer matrix R, is reduced to: 
\begin{equation}
	R_q = 
	\begin{bmatrix}
	\frac{f_2 f_4}{f_1 f_3} & 0 \\
	0 & \frac{f_1 f_3}{f_2 f_4}	
	\end{bmatrix}
\end{equation}

\nrnote{add detail about this matrix comes from and expand matrix to x and y}
Where $f_1 \ldots f_4$ stand for the focal lengths of each quad before the kicker. 
Due to other experiments in the AWA tunnel, 
the first quadrupole was required to be at least $\SI{3}{m}$ away from the exit of the 
last accelerating cavity in the linac. This gives the initial drift length and value
for $f_1$. 
 
 
 
 
 
\iffalse 
 \[
 \begin{bmatrix}
 x_{11}       & x_{12} & x_{13} & \dots & x_{1n} \\
 x_{21}       & x_{22} & x_{23} & \dots & x_{2n} \\
 \hdotsfor{5} \\
 x_{d1}       & x_{d2} & x_{d3} & \dots & x_{dn}
 \end{bmatrix}
 =
 \begin{bmatrix}
 x_{11} & x_{12} & x_{13} & \dots  & x_{1n} \\
 x_{21} & x_{22} & x_{23} & \dots  & x_{2n} \\
 \vdots & \vdots & \vdots & \ddots & \vdots \\
 x_{d1} & x_{d2} & x_{d3} & \dots  & x_{dn}
 \end{bmatrix}
 \]
 \fi