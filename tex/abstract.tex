\documentclass{article}
\usepackage{setspace}
\doublespacing
\setlength\parindent{1.27cm}
\pagenumbering{gobble}

\usepackage[left=1.5in,right=1in,top=2in,bottom=1in]{geometry}

\begin{document}
\noindent\uppercase{Beam Line Design for Fully Staged Two Beam Acceleration at the Argonne Wakefield Accelerator Facility}


\noindent Nicole Neveu, Ph.D.

\noindent Illinois Institute of Technology, December 2018 

\noindent Adviser: Dr. Linda Spentzouris \\



	Two beam acceleration (TBA) is a candidate for future high energy physics machines and FEL user facilities. 
	This is a scheme in which an electron accelerator uses a ``drive'' beam to transport and supply the RF power needed for acceleration
	on a secondary and independent 'witness' accelerator. This technology is attractive for its potential to 
	improve the efficiency and simplicity of large scale machines. At the Argonne Wakefield Accelerator Facility (AWA), 
	research into this potential accelerator scheme is ongoing. Completed experiments include a simplified staging set up, 
	where high-charge, 65 MeV drive bunch trains were injected from the RF photoinjector into decelerating 
	structures to generate a few hundred MW's of RF power. This RF power was transferred through an 
	RF waveguide to accelerating structures that were used to accelerate the witness beam. 
	Staging refers to the sequential acceleration (energy gain) in two or more structures on the witness beam line. 
	
	The main limitation in past experiments was difficulty achieving 100\% transmission in the 
	second stage which resulted in lower power generation. 
	AWA plans to demonstrate fully staged TBA, which requires a separate beam line for each decelerating/accelerating pair.
	In this thesis, design specifications and initial hardware tests  needed for a new, independent beam line for TBA was done. 
	Simulations of the drive line were done using the code OPAL.
	Since OPAL was new to the AWA group, a benchmark comparison with ASTRA and GPT was done to validate initial results. 
	Then two optimization algorithms were investigated and used to optimize the drive line at 40 nC.
	Comparison of results between the two algorithms were done, with no major discrepancies found.
	Then large scale and parallel optimizations were done for the optics configuration in the fully staged TBA beam line design.  
	
	A kicker was designed and incorporated into the drive beam line to accomplish a modular design so that 
	each accelerating structure can be independently powered by a separate drive beam. 
	Experimental measurements of the kicker indicate the angle increases linearly with the supplied voltage, 
	and the angle achieved meets the design requirements for fully staged TBA. 
	Optics optimization was done to minimize the beam size at the center of the 
	decelerating structures to ensure good charge transmission. 
	The resulting design will be the basis for proof of principle experiments that will take place at the AWA facility. 

\end{document}