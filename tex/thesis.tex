
%%%%%%%%%%%%%%%%%%%%%%%%%%%%%%%%%%%%%%%%%%%%%%%%%%%%%%%%%%%%%%%%%%%%%%%%%%%%%%%
%
% Nicole's Thesis, IIT 2018
%
%%%%%%%%%%%%%%%%%%%%%%%%%%%%%%%%%%%%%%%%%%%%%%%%%%%%%%%%%%%%%%%%%%%%%%%%%%%%%%%
\documentclass{iitthesis}

% Document Options:
%
% Note if you want to save paper when printing drafts,
% replace the above line by
%
%   \documentclass[draft]{iitthesis}
%
% See Help file for more about options.

\usepackage[dvips]{graphicx}    % This package is used for Figures
\usepackage{rotating}           % This package is used for landscape mode.
\usepackage{epsfig}
\usepackage{subfigure}          % These two packages, epsfig and subfigure, are used for creating subplots.
% Packages are explained in the Help document.


\begin{document}

\title{Staged Dielectric Two Beam Acceleration at the Argonne Wakefield Accelerator Facility}

\author{Nicole Neveu }
\degree{Doctor of Philosophy}
\dept{Physics}
\date{May 2018}
\copyrightnoticetrue      % crate copyright page or not
%\coadvisortrue           % add co-advisor. activate it by removing % symbol to add co-advisor
\maketitle                % create title and copyright pages


\prelimpages         % Settings of preliminary pages are done with \prelimpages command

%%%  Acknowledgement %%%
\begin{acknowledgement}     % acknowledgement environment, this is optional
	\par  Family, Lalo, Linda, John, AWA group
	% or \input{acknowledgement.tex} % you need a separate acknowledgement.tex file to include it.
\end{acknowledgement}

% Table of Contents
\tableofcontents
\clearpage

% List of Tables
\listoftables

\clearpage

%List of Figures
\listoffigures

\clearpage

%List of Symbols(optional)

\listofsymbols
\SymbolDefinition{$c$}{Speed of Light}
\SymbolDefinition{$\epsilon$}{Dielectric Permittivity}
\SymbolDefinition{$\epsilon$}{6D Phase Space Emittance}
\SymbolDefinition{$\gamma$}{Relativistic Kinetic Energy}

\clearpage

%%% Abstract %%%
\begin{abstract}           % abstract environment, this is optional
	\par Staged two beam acceleration using dielectric structures was achieved
	at the Argonne Wakefield Accelerator. In this thesis, I discuss the beam 
	line design and simulation, followed by experimental results of said
	beam line.  
\end{abstract}


\textpages     % Settings of text-pages are done with \textpages command

\Chapter{INTRODUCTION}

\Section{Motivation} \label{sec:int}

\Section{Argonne Wakefield Accelerator Facility}

\Section{Dielectric Structures}

\Subsection{Power Generation}

\Subsection{Accelerating Structures}

\Section{Design Requirements}

% An example for enumerate
\begin{enumerate}
	\item Kicker Design
	\item Septum Design
	\item Optimization 
	mainly muscle
\end{enumerate}

% A quotation example
% Every quota must be accompanied by a reference to the source
% in a footnote or in the Bibliography
\begin{quotation}
	test
\end{quotation}

\clearpage

\Chapter{Beam Dynamics}

\Section{Code and Resources Used}

\Section{Optimization Techniques}

\Section{Simulation Results}

\footnote{My Footnote} 


\Chapter{Experimental Results}

\Section{Kicker Installation}

\Section{Septum Installation} 

\Section{Measurement of Single Stage Power Extraction}

\Section{Measurement of Single Stage Accelerating Gradient}

\Section{Measurement of Staged Two Beam Acceleration}


\Chapter{CONCLUSION}
%   \input{Conclusion.tex}
You need a Conclusion.tex file


\Section{Summary}


This was just to create a sample section...

\clearpage


%
% APPENDIX
%
\appendix

\Appendix{Calculations?}

......

\moretox

\Appendix{OPAL stuff? MCS Stuff?}

Your second appendix text....

\Appendix{Name of your Third Appendix}

Your third appendix text....
%
% BIBLIOGRAPHY
%
% you have two options: 1) create bibliography manually,
% 2) create bibliography automatically. See BibliographyHelp.pdf file for details.


\bibliographystyle{plain}
\bibliography{mybib}

\end{document}  % end of document





























