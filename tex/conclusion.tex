\Chapter{CONCLUSION}

Two beam acceleration (TBA) is a promising candidate for future high energy machines. Active research and development on this topic is taking place at the AWA. There have already been successful demonstrations of single stage metallic and dielectric 
TBA at the AWA. Further, a simplified two stage scheme was also demonstrated, with both bunch trains passing through both decelerating structures. 
The last step is to demonstrate a fully staged TBA scheme, where each decelerating structure is driven by an independent bunch train.  In order to achieve this, two beamlines are required, each with one decelerating structure.   A kicker is used to route one bunch train into a second beamline. The work presented supports fully staged TBA by optimizing the design of the new beamline, including a determination of the adjustments needed to the kicker hardware.  The beamline simulation and optimization were done using the OPAL code on a high performance computing platform.  Different optimization techniques were compared, and a solution for the beamline design was found that achieves 100\% transmission of beam with good power generation through the decelerating structure. 

Initial experimental work included improving the ultraviolet pulse train. This \lsnote{was \sout{one}} done through careful measurement of the UV optics on the multisplitter table. Results showed the current splitters were of \lsnote{insufficient \sout{low}} quality, and a higher grade set was ordered. Measurements of the new splitters \lsnote{\sout{also}} took place 
to select the best \lsnote{combination for producing a uniform bunch train \sout{out of the order}}. 
A considerable improvement in the UV pulse train intensity distribution was observed. The deviation between pulses was reduced by about 16.5\%.
Next RF power measurements were done for each of the accelerating 
cavities at the AWA. A pick up probe in the cavity sends a signal back to the control room when the accelerator is running. A power meter was used to measure the signal and the cables were calibrated to account for losses.  These measurements helped guide simulation parameters. \lsnote{A solenoid scan was done using the data analysis script
and helped determine a source of discrepancy between experiment and 
simulations.} 

\lsnote{The beam properties must be determined for good understanding of beam generation and transport, as well as for comparison to simulation results.}  Several sets of energy measurements were taken. \lsnote{\sout{ for the same reason.} A dipole spectrometer was used \sout{
The method}} for measuring energy \lsnote{\sout{included using a dipole spectrometer. }}
When the beam is bent \lsnote{by} a known angle, the energy can be \lsnote{\sout{back}} calculated. 
Probably the most important beam diagnostic at the AWA, 
because of their ease to use are the Yittrium Aluminum Garnet (YAG) screens. 
Cameras pointed at the YAG screens can take pictures of the 
transverse beam profile. 
Various data analysis techniques can then be used to determine 
the beam size and charge. A script was written in Python
to perform this analysis, and it is available on github for others to use. 
\lsnote{\sout{A solenoid scan was done using the data analysis script
and helped confirm discrepancies between experiment and 
simulations.}} 
The last beam diagnostic used in this work was a Coherent Transition Radiation (CTR) screen. When the beam passes
through a specially designed foil in the beam line, 
it emits CTR. This light can be focused into a Michelson 
interferometer and used to calculate the bunch length. 
This is a difficult measurement \lsnote{due \sout{do}} to the necessity of \lsnote{using} a 
bolometer to measure intensity. 

\lsnote{The simulation work began with \sout{Moving on to simulation work,}} a benchmark \lsnote{\sout{was done}} between OPAL, ASTRA, and GPT 
to confirm the correct use of OPAL. 
Part of that work included convergence studies needed for computational parameters such as number of particles and grid size.
Bunch length simulations were compared to the experimental 
measurements. \lsnote{\sout{done previously.}}
\lsnote{\sout{Further work included optimization of the drive line for 40 nC bunches.}} 
A model based optimization method was used first, 
and compared to a genetic algorithm. 
Both methods returned similar results, indicating either
can be used for photoinjector optimization.  \lsnote{Once the OPAL and the optimizaiton techniques were well understood,} the code was used to do the optimization of the new drive line for \SI{40}{nC} bunches.  \lsnote{Particular attention was paid to the transverse beam size and bunch length at the entrance of the decelerating structure.  This structure is the smallest aperture in the beamline.  The bunch length affects the magnitude of power generation achievable in the structure.  There is a trade-off in achieving small transverse emittance and short bunch length.  The optimization produced a Pareto front whereby that trade-off could be understood, and the minimum combinations determined.}

A fast rise time kicker \lsnote{design was modified \sout{was designed,} and then} fabricated and tested 
at the AWA. 
The kicker meets specifications and the deflection angle 
along with mechanical constraints were used to layout 
a fully staged TBA beam line. 
Massively parallel optimizations were then done to 
determine an optics configuration that would guarantee 
100\% transmission at the PETS location. 
A 2D solution was found that did not require use of the
last \lsnote{quadrupole focusing} triplet in the beam  line. 
Differences in the 2D and 3D simulations indicate the optics 
solution needed to be adjusted to account for coherent syncrotron radiation (CSR) and 3D field maps. 
Since the 2D solution did not require hard focusing in any of the magnets, 
there \lsnote{was \sout{is}} plenty of \lsnote{range \sout{room}} to optimize the optics for the 3D scenario.
In conclusion, a beam line design was \lsnote{achieved \sout{presented}} that is capable of 
demonstrating fully staged TBA. 










