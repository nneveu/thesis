\documentclass[letterpaper,  %a4paper
               %boxit,
               %titlepage,   % separate title page
               %refpage      % separate references
              ]{jacow}
%
% CHANGE SEQUENCE OF GRAPHICS EXTENSION TO BE EMBEDDED
% ----------------------------------------------------
% test for XeTeX where the sequence is by default eps-> pdf, jpg, png, pdf, ...
%    and the JACoW template provides JACpic2v3.eps and JACpic2v3.jpg which
%    might generates errors, therefore PNG and JPG first
%
\makeatletter%
	\ifboolexpr{bool{xetex}}
	 {\renewcommand{\Gin@extensions}{.pdf,%
	                    .png,.jpg,.bmp,.pict,.tif,.psd,.mac,.sga,.tga,.gif,%
	                    .eps,.ps,%
	                    }}{}
\makeatother

% CHECK FOR XeTeX/LuaTeX BEFORE DEFINING AN INPUT ENCODING
% --------------------------------------------------------
%   utf8  is default for XeTeX/LuaTeX 
%   utf8  in LaTeX only realizes a small portion of codes
%
\ifboolexpr{bool{xetex} or bool{luatex}} % test for XeTeX/LuaTeX
 {}                                      % input encoding is utf8 by default
 {\usepackage[utf8]{inputenc}}           % switch to utf8

\usepackage[USenglish]{babel}			 

\usepackage[final]{pdfpages}
\usepackage{multirow}
\usepackage{ragged2e}
\usepackage{tikz}
\usetikzlibrary{shapes.misc}
\usetikzlibrary{shapes,arrows,decorations.markings,shadows,positioning}

%
% if BibLaTeX is used
%
\ifboolexpr{bool{jacowbiblatex}}%
 {%
  \addbibresource{jacow-test.bib}
  \addbibresource{biblatex-examples.bib}
 }{}
\listfiles

%
% command for typesetting a \section like word
%
\newcommand\SEC[1]{\textbf{\uppercase{#1}}}
\newcommand{\lsnote}[1]{\textsf{{\color{violet}{ LS note:}   #1 }}}
\newcommand{\nnnote}[1]{\textsf{{\color{blue}{ NN note:}   #1 }}}
\newcommand{\jlnote}[1]{\textsf{{\color{green}{ JL note:}   #1 }}}
\usepackage{floatrow}
\floatsetup[table]{capposition=top}

%%
%%   Lengths for the spaces in the title
%%   \setlength\titleblockstartskip{..}  %before title, default 3pt
%%   \setlength\titleblockmiddleskip{..} %between title + author, default 1em
%%   \setlength\titleblockendskip{..}    %afterauthor, default 1em

%\copyrightspace %default 1cm. arbitrary size with e.g. \copyrightspace[2cm]

% testing to fill the copyright space
%\usepackage{eso-pic}
%\AddToShipoutPictureFG*{\AtTextLowerLeft{\textcolor{red}{COPYRIGHTSPACE}}}

\begin{document}

\title{Photoinjector Optimization using a Derivative-Free, Model-Based, 
	Trust-Region Algorithm for the Argonne Wakefield Accelerator } 
	
	%\NoCaseChange{JACoW} conferences\thanks{Work supported by ...}}

\author{N. R. Neveu\textsuperscript{1}\thanks{nneveu@hawk.iit.edu}, J. Larson, J. G. Power, Argonne National Laboratory, Lemont, USA \\
		L. K. Spentzouris, \textsuperscript{1}Illinois Institute of Technology, Chicago, USA}
	
\maketitle

%
\begin{abstract}
% ----------------------------------------------------
Model-based, trust-region, derivative-free algorithms 
are increasingly popular for optimizing computationally 
expensive numerical simulations. A strength of such
methods is their efficient use of function evaluations. 
In this paper, we use these algorithms to optimize 
the beam dynamics in two cases of interest at the 
Argonne Wakefield Accelerator (AWA) facility. 
First, we minimize the emittance of the electron 
bunch produced by the AWA drive rf photocathode gun 
alone by adjusting three parameters: rf gun phase, 
solenoid strength, and laser radius. The algorithm 
used converges to a set of parameters with an
emittance of 1.08 mm-mrad. Second, we expand 
the number of optimization parameters to model 
the complete AWA rf photoinjector linac 
(the gun and six accelerating cavities). 
These results are used in a Pareto study of the 
trade-off between beam emittance and bunch 
length for the AWA linac.
\end{abstract}


\section{Introduction}
% ----------------------------------------------------


\section{Optimization Algorithms}
% ----------------------------------------------------
BOBYQA \cite{bobyqa}. . .


\section{Code}
% ----------------------------------------------------
To simulate beam dynamics with 3D space charge, 
the open source particle-in-cell code OPAL-T 
\cite{opal} was used. To implement the techniques 
described in the previous section, an open source 
%abstract?
python package called nlopt \cite{nlopt} was used
in combination with python code written at ANL for this project.
\nnnote{Should we add where the code is at? -  
	All progress is version controlled using xgitlab.
Interested parties can pull the code from the following repo: 
git@xgitlab.cels.anl.gov:jmlarson/emittance\_minimization.git}

\section{AWA Facility }
% ----------------------------------------------------
% A beam line used for production of high charge electron bunches at the Argonne Wakefield Accelerator (AWA) was used..  Refer to your table, `typical run parameters are given in table' 
A beam line installed at the AWA was used as the 
simulation model for this study. The model consists 
of a 1.5 cell rf gun paired with three solenoids: buck, 
focusing, and matching.  
The gun is followed by six rf accelerating cavities. 
See Figure \ref{fig:beamline} for the beam line layout. 
The gun and cavities are operated at \SI{1.3}{GHz}. 
A CsTe cathode and UV laser are used to generate 
the electrons.
\def \gunleft {-1.2}
\def \gunright {-0.3}

%This should say "left", but I'm lazy and didn't change it
\def \loneright {1.0}
\def \ltworight {3.5}
\def \lthreeright {5.0}
\def \lfourright {7.0}
\def \lfiveright {8.5}
\def \lsixright {10}
\begin{figure*}
	\centering
	\begin{center}
		\begin{tikzpicture}
		%Gun drawings
		\draw[fill=orange, very thick, rounded corners =0.3cm] (\gunleft,0.0)rectangle (\gunright,2) node[pos=.5, white] {\textbf{Gun}} ;
		\draw[ultra thick, fill=black!60!green] (\gunleft-0.45,-1)rectangle  (\gunright-0.45,0) node[pos=.5, white] {\textbf{$S_1, S_2$}} ;
		\draw[ultra thick, fill=black!60!green] (\gunleft-0.45,2)rectangle  (\gunright-0.45,3) node[pos=.5, white] {\textbf{$S_1, S_2$}} ;
		\draw[ultra thick, fill=black!60!green] ({\gunright+0.3},0)rectangle  (0.5,2) node[pos=.5, white] {\textbf{$S_3$}} ;
		
		%Linac drawings 
		\draw[fill=blue, ultra thick, rounded corners =0.2cm] (\loneright,0)rectangle  ({\loneright+1},2) node[pos=.5, white] {L1} ;
		\draw[fill=blue, ultra thick, rounded corners =0.2cm] (\ltworight,0)rectangle  ({\ltworight+1},2) node[pos=.5, white] {L2};
		\draw[fill=blue, ultra thick, rounded corners =0.2cm] (\lthreeright,0)rectangle ({\lthreeright+1},2) node[pos=.5, white] {L3};
		\draw[fill=blue, ultra thick, rounded corners =0.2cm] (\lfourright,0)rectangle ({\lfourright+1},2) node[pos=.5, white] {L4};
		\draw[fill=blue, ultra thick, rounded corners =0.2cm] (\lfiveright,0)rectangle ({\lfiveright+1},2) node[pos=.5, white] {L5};
		\draw[fill=blue, ultra thick, rounded corners =0.2cm] (\lsixright,0)rectangle ({\lsixright+1},2) node[pos=.5, white] {L6};
		
		\draw[very thick] (\gunleft,-1.5) -- (14,-1.5);
		%\path [draw=black, fill=black] (1,-2.5) circle (2pt); %black circle
		%\path [draw=black, fill=white, thick] (2,-2.5) circle (2pt); %white circle
		\draw[latex-latex] (\gunleft,-1.5) -- (14,-1.5) ; 
		\foreach \x in  {0.5, 1.0, 3.5, 5.0, 7.0, 8.5, 10, 12.5} %tick marks
		\draw[shift={(\x,-1.5)},color=black] (0pt,3pt) -- (0pt,-3pt);
		\foreach \x in {0.5, 1.0, 3.5, 5.0, 7.0, 8.5, 10, 12.5}
		\draw[shift={(\x,-1.7)},color=black] (0pt,0pt) node[below] 
		{$\x$};
		
		\node[draw, fill=yellow, star, star points=5, star point ratio=0.6, minimum size=0.6cm]
		at (12.5,1.0) {$z_1$};
		\end{tikzpicture}
	\end{center}
\caption{Layout of the portion of the AWA beam line used as simulation model.}
% modeled by the simulation
\label{fig:beamline}
\end{figure*} 


\section{Gun Optimization}
% ----------------------------------------------------
% A little more info here
Much work has been done to optimize 1.5 cell rf guns
at \SI{1}{nC} \cite{pitz}. This known solution 
was used as a test 
of the optimization algorithm BOBYQA \cite{bobyqa}. 
A single objective ($\epsilon_x$) optimization was 
done over a length of \SI{5}{m}, 
and all linacs were turned off in these simulations. 
Non-varying parameters were based on work done at 
PITZ \cite{pitz}, and a benchmark of OPAL-T, GPT, and ASTRA 
using the AWA's drive gun \cite{benchmark}, see Table \ref{tab:gun}.
\begin{table}[hbt] %or [hbt] ?
	\centering
	\begin{tabular}{l c c}
		%\toprule
		\textbf{Parameter} & \textbf{Value} \\
		\hline %\midrule
		Charge  & \SI{1}{nC} \\
		Gradient & \SI{60}{MV/m} \\
		Laser FWHM & \SI{20}{ps} \\
		Laser Rise and Fall Time & \SI{6}{ps} \\
		Kinetic Energy at Cathode  & \SI{0.55}{eV} \\
		Buck and Focusing Solenoids & \SI{550}{A}
		%\bottomrule
	\end{tabular}
	\caption{Non-varying Simulation Parameters for Gun Optimization Runs}
	\label{tab:gun}
\end{table}
Three parameters were variable: matching solenoid strength ($S_3$), gun phase
($\phi_g$), and laser radius (R). Note, phase 
refers to the difference between injection phase and 
phase of max energy gain (on crest). The range for each 
parameter is shown in Table \ref{tab:parameters}. 
The laser full width half max, T, and cavity phases, $\phi_l$, were added in the 
Pareto front simulations described later. For the 
remainder of the paper, the variables in Table \ref{tab:parameters} 
will be referred to as $v=\left[S_3, \phi_g, R, T, \phi_l\right]$, where 
$\phi_l=[\phi_{l_1},\,\phi_{l_2},\,\phi_{l_3},\,\phi_{l_4},\,\phi_{l_5},\,\phi_{l_6}]$
represents the phase of each linac cavity 1-6. 
\nnnote{IPAC format - No period for table headers, and capital for each word.
Figure captions should be like a sentence w/ period, only first word capitalized.}
\begin{table}[hbt] %or [hbt] ?
	\centering
	\begin{tabular}{ l *{3}{c}}
		%\toprule
		\textbf{Variable} & \textbf{Range} & \textbf{Unit} \\
		\hline %\midrule
		Solenoid Strength & $ 0 \le S_3 \le 440$  & A \\
		Phase of Gun & $-100^\circ \le \phi_g \le 100^\circ$  & Degrees \\
		Laser Radius & $0.003 \le R \le 0.015$  & m \\
		Laser FWHM & $2 \le T \le $10  & ps \\
		Cavity Phase & $-20 \le \phi_l \le 20$  & MV/m \\
		%\bottomrule
	\end{tabular}
	\caption{Variable Simulation Parameters for Gun and Linac Optimization Runs}	
	\label{tab:parameters}
\end{table}

% More clarity needed different (typical operating?) points - and what is best sampled point?
Optimization runs were started from five different points that 
included the best known point and best sampled point
among others. All runs converged to an emittance of 
$\SI{1.08}{\um}$ in less than 100 simulation evaluations. 
An exhaustive search of the parameter space was not done, 
so there may be local minima that were not found.
However, the results match expectations based on literature.

\section{Linac Optimization} 
% ----------------------------------------------------
Next, the gun and linac as shown in Figure \ref{fig:beamline}, 
was optimized over ten variable parameters, see Table \ref{tab:parameters}, 
and two objectives: emittance, $\epsilon_x$, and bunch length, $\sigma_z$. 
The location of interest is $z_1=\SI{12.51}{m}$, 
as this is the entrance of the first 
quad after the last accelerating cavity. 
We choose to optimize $\epsilon_x$
instead of $\epsilon_{xy}$, because 2D field maps were 
used for the rf cavities, with no appreciable 
difference between the resulting x and y values.

The non-varying parameters for all 
linac simulation runs are shown in Table \ref{tab:linac}.
The initial kinetic energy parameter is missing because a different
emission model was used in this case. OPAL-T has the 
ability to model emission based on cathode material and laser
properties. This method was used to simulate emission
from a CsTe cathode using a laser with initial kinetic energy of 4 eV. 
These are typical operating conditions at AWA. 
\begin{table}[hbt] %or [hbt] ?
	\centering
	\begin{tabular}{l c c}
		%\toprule
		\textbf{Parameter} & \textbf{Value} \\
		\hline %\midrule
		Charge  & \SI{40}{nC} \\
		Laser Rise and Fall Time & \SI{1.0}{ps} \\
		Gun Gradient & \SI{70}{MV/m} \\
		Buck and Focusing Solenoids & \SI{550}{A}\\
		Cavity Gradient 1-4 & \SI{25}{MV/m} \\
		Cavity Gradient 5-6 & \SI{27}{MV/m} \\
		
		%\bottomrule
	\end{tabular}
	\caption{Non-varying Simulation Parameters for Linac Optimization Runs}
	\label{tab:linac}
\end{table}

A 1,000 point random sample of the linac parameter 
space was generated and evaluated. Out of these points, 132 
simulation evaluations survived without loosing particles
% losing 
or crashing. From this set, the emittance and bunch length of each 
surviving evaluation at $z_1=\SI{12.51}{m}$ was scaled to a unit cube, 
as shown in Equation \ref{eq:scale}, so
that the two objectives could be weighed fairly. 
\begin{equation}
f(v,z_1) = \frac{f_0(v,z_1)-f^-(v,z_1)}{f^+(v,z_1)-f^-(v,z_1)} 	
\label{eq:scale}
\end{equation}
Where $f_0(v,z_1)$ is either $\epsilon_x(v,z_1)$ or $\sigma_z(v,z_1)$,
$f^-(v,z_1)$ is the minimum value in the set, 
$f^+(v,z_1)$ is maximum value in the set, and $f(v,z_1)$ is the scaled value.  
The scaled values were then used in combination with eleven weights, $w_1$ and $w_2$,
to generate a set of new objectives, as shown in Eq. \ref{eq:newobj}. 
\begin{equation}
\begin{aligned}
	w_1 \in \left\{ 0.0,\,0.1, \,0.2 \ldots 1.0 \right\}\\
	w_2 = 1 - w_1
\end{aligned}
\end{equation} 
 
\begin{equation}
\text{minimize } \, \, w_1 \,\epsilon_x(v,z_1) + w_2 \,\sigma_z(v,z_1)
\label{eq:newobj}
\end{equation}
The points that corresponded to the minimum $f(v,z_1)$ per weight
were then used as starting points for BOBYQA optimization
runs. Six unique starting points resulted in all minimum. 
For example, sample number 18 resulted
in the minimum for weights 5, 6, and 7.

\section{Pareto Front for Linac} 
% ----------------------------------------------------
Figure XXXX shows the sample points, 
optimized values, and resulting Pareto front.


\section{Conclusion}
% ----------------------------------------------------
% I thought you said once that you tried the gun optimization parameters in the real machine?
% If so, this ought to be included.
Using an AWA beam line as the model, BOBYQA
was used to optimize the gun and linac. A Pareto front 
comparing the trade-off between bunch length and emittance
was generated for the linac. Future work will include
experimental measurements to verify the Pareto front, 
comparison to genetic algorithms, and implementation of BOBYQA 
in the optimization tool opt-Pilot \cite{optpilot}.

\section{Acknowledgment}
% ----------------------------------------------------
We gratefully acknowledge the computing resources
provided on Blues, a high-performance computing cluster
operated by the LCRC at Argonne National Laboratory.
This material is based upon work supported by the 
U.S. Department of Energy, Office of Science, under 
contract numbers DE-AC02-06CH11357, DE-AC02-06CH11357, 
and grant number DE-SC0015479. 

% ----------------------------------------------------
\bibliographystyle{plain}
\bibliography{../../bibs/masterbib}
\end{document}
	