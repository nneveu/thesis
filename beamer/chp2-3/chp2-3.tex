%%%%%%%%%%%%%%%%%%%%%%%%%%%%%%%%%%%%%%%%%%%%%%%%%%%%%%%%%%%%%%%%%%%%%%
%%%%%%%%% Select one of the options, and comment the rest of them

%%%%%%%%%% Option 1:  to compile with pdflatex : parameter "t" - to align to the top
\documentclass[professionalfonts,t]{beamer}
%sans font?

%%%%%%%%%% Option 3: to create handout for print
%\documentclass[t,handout]{beamer}
%\usepackage{pgfpages}              % to put several slides on one page
%\pgfpagesuselayout{2 on 1}[a4paper, border shrink=5mm]             % 2 slides on 1 page
%\pgfpagesuselayout{4 on 1}[a4paper,landscape, border shrink=5mm]   % 4 slides on 1 page, and landscaped


%%%%%%%%%%%%%%%%%%%%%%%%%%%%%%%%%%%%%%%%%%%%%%%%%%%%%%%%%%%%%%%%%%%%
%%%%%%%%%%%%%% Select the Theme %%%%%%%%%%%%%%%%%%%%%%%%%%%%%%%%%%%
\usetheme{Dresden}     % OK
%\usetheme{Berlin}
%\usetheme{Bergen}      % NO
%\usetheme{Boadilla}    % NO
%\usetheme{Copenhagen}  % NO
%\usetheme{Hannover}    % NO
%\usetheme{Luebeck}     % NO
%\usetheme{Marburg}     % NO
%\usetheme{Pittsburgh}  % NO
%\usetheme{default}
%\usetheme{Singapore}   % OK
%\usetheme{boxes}
%\usecolortheme{structure}
%\usecolortheme{rose}
%\usecolortheme{beaver}


\definecolor{mymaroon}{cmyk}{0.0, 1.0, 1.0, 0.498}
\definecolor{myblue}{cmyk}{1.0, 1, 0, 0.5}
\definecolor{mygreen}{cmyk}{100, 0, 100, 50}
\setbeamercolor*{palette secondary}{use=structure,fg=white,bg=myblue}
\setbeamercolor*{palette tertiary}{use=structure,fg=white,bg=mymaroon}

%\usepackage{beamerthemesplit}              %
\beamertemplateballitem % fancy bullets and numbering

\setbeamertemplate{navigation symbols}{}   % suppress navigation symbols
\addtobeamertemplate{frametitle}{}{%
	\logo{../../images/IIT_logo}
	\iffalse
	
	\begin{tikzpicture}[remember picture,overlay]
	\node[anchor=center, yshift=-13pt, xshift=-5pt] at (current page.north) 
	{\includegraphics[height=1.1cm]{../images/Argonne_cmyk_black-eps-converted-to}\hspace{10cm}};
	
	\node[anchor=north east, yshift=3pt, xshift=0pt] at (current page.north east) 
	{\includegraphics[height=0.7cm]{../images/IIT_Logo_blk}};
	\end{tikzpicture}
    
     \fi
}
% other possibilities to include LOGO. it puts it in RLC

%
%\pgfdeclareimage[width=1cm]{logo}{../images/IIT_Logo}
%\logo{\pgfuseimage{logo}}


% load additional packages

\usepackage{xcolor}
\usepackage{graphicx}
\usepackage{amsmath}
\usepackage{amssymb}
\usepackage{amsthm}
\usepackage{graphicx}
\usepackage{url}
\usepackage{color}
\usepackage{booktabs} % Allows the use of \toprule, \midrule and \bottomrule in tables
\usepackage{pifont}% http://ctan.org/pkg/pifont
\usepackage{epstopdf}
\usepackage[export]{adjustbox}
\usepackage{tikz}
\usetikzlibrary{shapes.misc}
\usetikzlibrary{shapes,arrows,decorations.markings,shadows,positioning}

% Your Abbreviations
\newcommand\bE{{\mathbb{E}}}
\newcommand\bR{{\mathbb{R}}}
\newcommand\bH{{\mathbf{H}}}
% End abbreviations

\newcommand\Wider[2][3em]{%
	\makebox[\linewidth][c]{%
		\begin{minipage}{\dimexpr\textwidth+#1\relax}
			\raggedright#2
		\end{minipage}%
	}%\textbf{}
}

%%%%%%%%%%%%%%%%%%%%% to edit the main text below
%NOTES ON SOME TECHNICS
%%%% Box %%%%%%%%%%%%%%%%%%%%%%%%%%%%%%%%%%%%%%%%%%%%%%%
%{\fbox{ \parbox[t]{10cm}{ SOME TEXT }}}

%%% include a picture. The file should be with extention EPS, e.g. FILENAME.EPS
%\begin{figure}[h]
%\centering
%\includegraphics[width=.7\linewidth]{FILENAME}
%\caption{{\footnotesize PUT_CAPTION }}
%\end{figure}

%\subtitle{}
%\institute[ANL/IIT]{Argonne National Laboratory\\Illinois Institute of Technology}

\title[June 2018]{Simulations, Optimization, and Experimental Measurements at the AWA}
\author[N.Neveu]{{\Large Nicole Neveu}}
\institute[ANL, IIT] % (optional, but mostly needed)
{   Illinois Institute of Technology \\
	Argonne National Laboratory \\
    \url{nneveu@anl.gov} 
}
% - Use the \inst command only if there are several affiliations.
% - Keep it simple, no one is interested in your street address.
\date{ \today \\
\includegraphics[width=3cm,keepaspectratio]{/home/nicole/Documents/presentations/logos/Argonne_cmyk_black}%
\hfill \hfill \hfill%
\includegraphics[width=4cm,keepaspectratio]{/home/nicole/Documents/presentations/logos/IIT_Logo_blk-eps-converted-to}%
}

%\date[IIT, April 2009]{
%           Space Charge 2017 \\ Oc 18, 2009  }



\begin{document}


\begin{frame}
  \titlepage
\end{frame}


\begin{frame}
	\frametitle{Outline}
	\tableofcontents
\end{frame}

\section{Simulations}
\subsection{AWA Facility Introduction}
\begin{frame}[plain]
\frametitle{Argonne Wakefield Accelerator Facility (AWA)}
\Wider[4em]{
	\setlength{\leftmargin}{0.1cm}
	
	%\begin{minipage}{0.6\textwidth}
		\begin{itemize}
			\item{Two photocathode guns and linacs}
			\begin{itemize}
				\item{\underline{\textbf{Drive Line}}: $Cs_2Te$ cathode, 6 linac cavities}
				\begin{itemize}
					\item{Charge 0.1-100nC}
					\item{Energy up to 70 MeV}
					
				\end{itemize}
				\item{\underline{\textbf{Witness Line}}: $Mg$ cathode, 1 linac cavity}
				\begin{itemize}
					\item{Charge 0.1-10nC}
					\item{Energy up to 15 MeV}
				\end{itemize}
			\end{itemize}
		\end{itemize}	
	%\end{minipage}
	%\begin{minipage}{0.35\textwidth}
	\vspace{0.5em}
		\centering
		\includegraphics[width=0.55\linewidth]{/home/nicole/Documents/presentations/space_charge_2017/drive_gun}
	%\end{minipage}
}
\end{frame}


\begin{frame}
\frametitle{AWA Facility}
Current experiments include:
\begin{itemize}
	\item{Two Beam Acceleration (TBA)}
	\item{Dielectric accelerating and decelerating structure tests}
	\item{Beam line design for TBA = my thesis}
\end{itemize}
\vspace{0.5cm}
\includegraphics[width=0.49\linewidth]{/home/nicole/Documents/presentations/space_charge_2017/stage}\hfill\includegraphics[width=0.49\linewidth]{/home/nicole/Documents/presentations/space_charge_2017/dielectrics}

\end{frame}

\begin{frame}
\frametitle{AWA Facility}
Current experiments include:
\begin{itemize}
\item{Emittance Exchange (EEX)}
\item{Electron Radiography Imaging (ERI)}
\item{Cathode Studies}
\item Plasma wakefield (very recent)
\end{itemize}
\vspace{0.3cm}
\centering
\includegraphics[width=0.65\linewidth]{/home/nicole/Documents/presentations/space_charge_2017/EEX}
\end{frame}


\subsection{Layout}
\begin{frame}
	\frametitle{Drive Line Layout}
	\begin{tikzpicture}[scale=\textwidth/20cm, text=black]
	%\begin{tikzpicture}[scale=0.5, text=black]
	\input{/home/nicole/Documents/thesis/beamer/tba_talk/bentbeam.tex}
	\end{tikzpicture}
	
	\vspace{3em}
	For the remainder of the talk, I will focus on simulation and experimental results for the beam line above.
\end{frame}

\begin{frame}
\frametitle{TBA Bent Beam Layout}
\begin{tikzpicture}[scale=\textwidth/20cm, text=black]
%\begin{tikzpicture}[scale=0.5, text=black]
\input{../tba_talk/bentbeam.tex}
\end{tikzpicture}

\vspace{-1em}
Mechanical considerations:
\begin{itemize}
	\item 1m between kicker and septum
	\begin{itemize}
		\item for separation $\ge$ 50mm in septum. 
	\end{itemize}
	\item 1.8m between septum and dipole
	\begin{itemize}
		\item for separation $\ge$ 0.5m of beam lines.
	\end{itemize}
	\item 15cm between quads for easy installation. 
	\item 0.3m between quads and PETS for yag screen.  
\end{itemize}
\end{frame}

\begin{frame}
	\frametitle{OPAL}
	Mention code I use here?
	Resources
\end{frame}


\begin{frame}
	\frametitle{Benchmarking}
	We start with the gun...
		Add gun benchmark pictures here
\end{frame}

\begin{frame}
	\frametitle{Early Linac Simulations}
	Linac and quads
\end{frame}


\section{Optimization}
\subsection{Model Based}
\begin{frame}
	\frametitle{Linac Optimization}
	Add BOBYQA work here
\end{frame}

\begin{frame}
	\frametitle{Linac Optimization cont..}
	BOBYQA work here
\end{frame}

\begin{frame}
	BOBYQA work here?
\end{frame}

\subsection{Genetic Algorithms}
\begin{frame}
\frametitle{GA Work}
Premise: 
\begin{itemize}
	\item Consider hardware constraints
	\item Assume no phase control
	\item Use brute force optimization to investigate parameter space.
\end{itemize}

\vspace{0.5em}
Given information above, does a nominal solution exist that will satisfy tba requirements for this non-ideal situation. With the requirements being:
\begin{itemize}
	\item 100\% transmission 
	\begin{itemize}
		\item good beam size at structure
		\item good emittance at structure
	\end{itemize}
	\item Reasonable bunch length at structure
	\begin{itemize}
		\item to maximize power generated
	\end{itemize}
\end{itemize}
	 
\end{frame}

\begin{frame}
\frametitle{TBA Optimization}
\vspace{-0.75em}
\begin{tikzpicture}[scale=\textwidth/26cm, text=black]
%\begin{tikzpicture}[scale=0.5, text=black]
\input{../tba_talk/bentbeam.tex}
\end{tikzpicture}
%\caption{TBA beam line layout at the AWA. The arrow at the end of each line indicates what direction the beam is traveling.
%PETS stands for Power Extraction and Transfer Structure, and ACC
%stands for Accelerating structure. The subscript index on each structure refers to which stage the structures belong to, first or second stage. }

\vspace{-1em}
First round, 13 design variables:

Simplest, and worst case scenario (no phase control).

These variables are swept during optimization runs.
\begin{table}[hbt] 
	\centering
	\begin{tabular}{ l *{3}{c}}
		\toprule
		\textbf{Variable} & \textbf{Range} & \textbf{Unit} \\
		\midrule
		Buck Focusing Sol. &  $ 50 \le S_1 \le 440$ & amps \\
		Matching Solenoid & $ 350 \le S_2 \le 500$  & amps \\
		Phase of Gun & $-30 \le \phi_g \le 0.0$  & degrees \\
		Laser FWHM & $1.5 \le T \le $10  & ps \\
		Quads$_{1-9}$ & $-8 \le Q_n \le 8$  & T/m \\
		\bottomrule	
	\end{tabular}
	
\end{table}
\end{frame}

\begin{frame}
\frametitle{Optimization Objectives}
Objectives are beam parameters the simulation tries to minimize.
These points are picked to improve beam parameters at key locations, 
i.e. the PETS structure.

\vspace{1em}
\begin{minipage}{0.45\textwidth}
	\begin{table}[hbt] 
		\centering
		\begin{tabular}{ l *{3}{c}}
			\toprule
			\textbf{Variable} &  \textbf{Unit} \\
			\midrule
			$\sigma_z$ 		& mm \\
			$\sigma_{x}$ 	& mm \\
			$\sigma_y$ 		& mm \\
			$\sigma_{px}$ 	& mm-mrad \\
			$\sigma_{py}$ 	& mm-mrad \\
			$dE$			& MeV\\
			\bottomrule	
		\end{tabular}	
	\end{table}
\end{minipage}
\begin{minipage}{0.45\textwidth}
	\begin{itemize}
		\item 3 optimized locations ($z_k$, $z_s$, $z_d$)
		\item 18 Objectives
		\item 10 variables 
		\item 6 Constraints
	\end{itemize}
\end{minipage}
\centering
Note, these take an extremely large time to simulate (order days).
\end{frame}




%%%%%%%%%%%%%%%%%%%%%%%%%%%%%%%%%%%%%%%%%%%%%%%%%%%%%%%%%%%%%%%%%%%%%%%%%%%%%%%%




\begin{frame}
\frametitle{PETS}
Reminder: form factor and bunch length are related. PETS aperture = 17.6 mm

\begin{minipage}{0.49\textwidth}
\includegraphics[width=\textwidth]{/home/nicole/Documents/awa-tba/whole_line/rms_z}
\end{minipage}
\begin{minipage}{0.49\textwidth}

\centering
\includegraphics[width=\textwidth]{/home/nicole/Documents/thesis_code/formfactorsqrd}
\end{minipage}
\end{frame}


\section{Experimental Measurements}


\begin{frame}
\frametitle{Sensitivity Analysis}

\vspace{-0.5em}
Justification for matching simulations to experiments...
Mismatch in energy, matching solenoid, or quads weakens beam results and prediction capabilities.

\centering
\includegraphics[width=0.8\textwidth]{/home/nicole/Documents/surrogatemodels/ml-ws-poster/awa-medium-o4/sensresults.pdf}
\end{frame}


\begin{frame}[t]
\frametitle{Matching Solenoid Scans}
\begin{columns}[T]
	\begin{column}{0.76\textwidth}
		\begin{minipage}{0.5\textheight}
			\includegraphics[width=1.0\linewidth]{/home/nicole/Documents/presentations/group_meetings/xbeamsizes_low_charge_sol_scan_11-02-2017}	\includegraphics[width=1.0\linewidth]{/home/nicole/Documents/presentations/group_meetings/ybeamsizes_low_charge_sol_scan_11-02-2017}
		\end{minipage}
		\begin{minipage}{0.5\textheight}
			\includegraphics[width=1.0\linewidth]{/home/nicole/Documents/presentations/group_meetings/xbeamsizes_high_charge_sol_scan_10-17-2017}
			\includegraphics[width=1.0\linewidth]{/home/nicole/Documents/presentations/group_meetings/ybeamsizes_high_charge_sol_scan_10-17-2017}
		\end{minipage}
	\end{column}
	\begin{column}{0.4\textwidth}
		%\vspace{-9em}
		\begin{itemize}
			\item only charge fluctuations are considered in error bars
			\item Need to check laser profiles  
			\item Radius probably needs to be adjusted in simulations of high charge
		\end{itemize}
	\end{column}
\end{columns}

\end{frame}
%%%%%%%%%%%%%%%%%%%%%%%%%%%%%%%%%%%%%%%%%%%%%%%%%%%%%%%%%%%%%%%%%%%%%%%%%%%%%%%%

%%%%%%%%%%%%%%%%%%%%%%%%%%%%%%%%%%%%%%%%%%%%%%%%%%%%%%%%%%%%%%%%%%%%%%%%%%%%%%%%
\subsection{Beam Size and Energy Measurements}
\begin{frame}
	\frametitle{Energy Measurments}
\end{frame}

\begin{frame}
	\frametitle{Beam Sizes}
\end{frame}


\subsection{Solenoid Scans}
\begin{frame}
\frametitle{BF Solenoid Scans}
\begin{columns}[T]
\begin{column}{0.76\textwidth}
	\begin{minipage}{0.5\textheight}
		\includegraphics[width=1.0\linewidth]{/home/nicole/Documents/presentations/group_meetings/xbeamsizes_0pt7nC_BFsol_scan2}
		\includegraphics[width=1.0\linewidth]{/home/nicole/Documents/presentations/group_meetings/ybeamsizes_0pt7nC_BFsol_scan2}%
	\end{minipage}
	\begin{minipage}{0.5\textheight}
		\includegraphics[width=1.0\linewidth]{/home/nicole/Documents/presentations/group_meetings/xbeamsizes_1nC_BFsol_scan}
		\includegraphics[width=1.0\linewidth]{/home/nicole/Documents/presentations/group_meetings/ybeamsizes_1nC_BFsol_scan}
	\end{minipage}
\end{column}
\begin{column}{0.4\textwidth}
	%\vspace{-9em}
	\begin{itemize}
		\item Again, radius may need adjustment
		\item M=0 images were partly off screen, need to play with fit for these
	\end{itemize}
\end{column}
\end{columns}
\end{frame}

%%%%%%%%%%%%%%%%%%%%%%%%%%%%%%%%%%%%%%%%%%%%%%%%%%%%%%%%%%%%%%%%%%%%%%%%%%%%%%%%
\begin{frame}
	\frametitle{Summary}
	\begin{itemize}
		\item Future simulation steps:
		\begin{itemize}
			\item Parametric study of effect of quad spacing on divergence
			\item Parametric study for relative quad strengths
			\item Full optimization with constraint based on above studies
			\item Re-optimize with phase control

		\end{itemize}
		\item Reminder: Near term proposed experiments don't require more hardware
		\item Beam studies needed to ensure success of TBA.
		
	\end{itemize}

	\vspace{1em}
	\centering
	\color{blue}\huge{Thanks for your attention!}

\end{frame}

%%%%%%%%%%%%%%%%%%%%%%%%%%%%%%%%%%%%%%%%%%%%%%%%%%%%%%%%%%%%%%%%%%%%%%%%%%%%%%%%
\end{document}
















